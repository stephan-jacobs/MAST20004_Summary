% This was prepared from Dave Richeson's excellent Latex Cheatsheet which can be found here: https://divisbyzero.com/2011/08/17/a-quick-guide-to-latex/
\documentclass[10pt,landscape]{article}
\usepackage{amssymb,amsmath,amsthm,amsfonts}
\usepackage{multicol,multirow}
\usepackage{calc}
\usepackage{ifthen}
\usepackage[landscape]{geometry}
\usepackage[colorlinks=true,citecolor=blue,linkcolor=blue]{hyperref}
\usepackage[x11names]{xcolor}
\usepackage{tcolorbox}



\ifthenelse{\lengthtest { \paperwidth = 11in}}
    { \geometry{top=.5in,left=.5in,right=.5in,bottom=.5in} }
	{\ifthenelse{ \lengthtest{ \paperwidth = 297mm}}
		{\geometry{top=1cm,left=1cm,right=1cm,bottom=1cm} }
		{\geometry{top=1cm,left=1cm,right=1cm,bottom=1cm} }
	}
\pagestyle{empty}
\makeatletter
\renewcommand{\section}{\@startsection{section}{1}{0mm}%
                                {-1ex plus -.5ex minus -.2ex}%
                                {0.5ex plus .2ex}%x
                                {\normalfont\large\bfseries}}
\renewcommand{\subsection}{\@startsection{subsection}{2}{0mm}%
                                {-1explus -.5ex minus -.2ex}%
                                {0.5ex plus .2ex}%
                                {\normalfont\normalsize\bfseries}}
\renewcommand{\subsubsection}{\@startsection{subsubsection}{3}{0mm}%
                                {-1ex plus -.5ex minus -.2ex}%
                                {1ex plus .2ex}%
                                {\normalfont\small\bfseries}}

\newcommand{\R}{\mathbb{R}}
\newcommand{\N}{\mathbb{N}}
\newcommand{\Z}{\mathbb{Z}}


\newtcbox{\highlight}[0]{boxsep=0pt,left=0pt,top=0pt,bottom=0pt,right=0pt,boxrule=0pt,arc=0pt,auto outer arc,colback=lightgray,width=6cm}

\makeatother
\setcounter{secnumdepth}{0}
\setlength{\parindent}{0pt}
\setlength{\parskip}{0pt plus 0.5ex}

% -----------------------------------------------------------------------

\begin{document}

\raggedright
\footnotesize

\begin{center}
     \Large{\textbf{MAST20004: Key Information Summary}} \\
\end{center}
\begin{multicols}{3}
\setlength{\premulticols}{1pt}
\setlength{\postmulticols}{1pt}
\setlength{\multicolsep}{1pt}
\setlength{\columnsep}{2pt}



%--------------------------------------------------------
%--------------------------------------------------------
%--------------------------------------------------------

\section{General Stuff}

\subsection{Integration by parts}

\begin{equation}
	\int u \frac{dv}{dx} dx = u \cdot v - \int v \frac{du}{dx} dx
\end{equation}

Choose $u$ and $\frac{dv}{dx}$ to make $v\frac{du}{dx}$ easy to integrate

\subsection{Geometric Series}

\begin{equation}
	S_n \equiv \sum_{k=0}^{n}r^k = \frac{1-r^{n+1}}{1-r}
\end{equation}

If $|r|<1$, $S_n$ converges as $n\rightarrow \infty$ and
\begin{equation}
	S_\infty = \sum_{k=0}^{n}r^k = \frac{1}{1-r} , \;\; |r|<1
\end{equation}

If sums start at $k=1$ instead of $k=0$:

\begin{equation}
	\sum_{k=1}^{n}r^k = \frac{r(1-r^n)}{1-r}
\end{equation}

\begin{equation}
	\sum_{k=1}^{n}r^k = \frac{r}{1-r}, \;\; |r|<1
\end{equation}


If sums start at some other value:

\begin{equation}
	\sum_{k=3}^{n}r^k = r^{3}\sum_{k=0}^{n}r^k
\end{equation}

%--------------------------------------------------------
%--------------------------------------------------------
%--------------------------------------------------------



\section{Set Operation Identities}

\begin{equation}
	A\cap (B\cup C)=(A\cap B) \cup (A\cap C)
\end{equation}

\begin{equation}
	A\cup (B\cap C)=(A\cup B) \cap (A\cup C)
\end{equation}

\subsection{De Morgan's Laws}
\begin{equation}
	(A\cup B)^{c} = A^{c}\cap B^{c}
\end{equation}

\begin{equation}
	(A\cap B)^{c} = A^{c}\cup B^{c}
\end{equation}



%--------------------------------------------------------
%--------------------------------------------------------
%--------------------------------------------------------



\section{Axioms of Probability}
$P(.)$ is a set function that maps $\mathcal{A}\rightarrow[0,1]$. $\mathcal{A}$ is the set of all events in $\Omega$, and $P$ maps those events onto the set of real numbers between 0 and 1.
\begin{enumerate}
	\item $P(A)\geq0$, for all events $A$.
	\item $P(\Omega)=1$
	\item \{\textcolor{red}{Countable Additivity}\} For any infinite sequence $\{A_1, A_2, A_3, ...\}$ of mutually exclusive/disjoint events, 
	\begin{equation}
		P\left(\bigcup_{i=1}^{\infty}A_i\right)=\sum_{i=1}^{\infty}P(A_i).
	\end{equation}
\end{enumerate}


%--------------------------------------------------------


\subsubsection{Properties of the Probability Function}

\begin{enumerate}
  \setcounter{enumi}{3}
  \item $P(\emptyset)=0$, since $\emptyset\cup\emptyset\cup\emptyset ... =\emptyset$
  
  \textcolor{SpringGreen4}{Note: This follows from Property 3.}
  
  \item \{\textcolor{red}{Finite Additivity}\} For any infinite sequence $\{A_1, A_2, A_3, ..., A_n\}$ of mutually exclusive/disjoint events, 
	\begin{equation}
		P\left(\bigcup_{i=1}^{n}A_i\right)=\sum_{i=1}^{n}P(A_i).
	\end{equation}
	
	\textcolor{SpringGreen4}{Note: This follows from Property 3.}
	
	\item $P(A^{c})=1-P(A)$, since $A\dot{\cup}A^{c}=\Omega$
	\item $A\subseteq B \Rightarrow P(A)\leq P(B)$, since $A\cup(A^{c}\cap B)=B$
	\item $P(A)\leq 1$, since $A\subseteq \Omega$
	\item \{\textcolor{red}{Addition Theorem}\}
	\begin{equation}
		P(A\cup B) = P(A)+P(B)-P(A\cap B)
	\end{equation}
	
	\item \{\textcolor{red}{Continuity}\}
	If either:
	\begin{enumerate}
		\item $A_1\subseteq A_2 \subseteq A_3 \subseteq ...$ AND $B=\bigcup_{i=1}^{\infty}A_i$, or
		\item $A_1\supset A_2 \supset A_3 \supset ...$ AND $B=\bigcap_{i=1}^{\infty}A_i$
	\end{enumerate}
	then:
	\begin{equation}
		\lim_{n\rightarrow \infty}{P(A_n)}=P(B)
	\end{equation}
\end{enumerate}



%--------------------------------------------------------
%--------------------------------------------------------
%--------------------------------------------------------


\section{Probability Identities}

\begin{equation}
	(B \subseteq A) \;\;\Rightarrow \;\; P(A \symbol{92} B) = P(A)-P(B)
\end{equation}

\begin{equation}
	P(A\cap B) \geq P(A)+P(B)-1
\end{equation}

\textcolor{SpringGreen4}{Note: This follows from Properties 8 and 9.}


%--------------------------------------------------------
%--------------------------------------------------------
%--------------------------------------------------------

\section{Random Variables}

\subsection{DEFINITION \textcolor{OrangeRed3}{Random Variable}}
\textbf{(this course only)}\\
\textsc{Consider a random experiment with sample space $\Omega$. A \textcolor{red}{function} $X$ which assigns to every outcome $\omega \in \Omega$ a real number $X(\omega)$ is called a \textcolor{red}{random variable}.}\\[5]

\textsc{The \textcolor{red}{state space} of $X$ is denoted $S_X\subseteq \R$}\\

%--------------------------------------------------------
% DISCRETE RANDOM VARIABLES
%--------------------------------------------------------

\subsection{DEFINITION \textcolor{OrangeRed3} {Discrete Random Variable}}

\textsc{A \textcolor{red}{discrete random variable} is one for which the set of possible values $S_X$ is \textbf{countable}. That is, $X$ can take only a countable number of values.}\\[5]

Such a variable generates a \textcolor{red}{partition} of the sample space into events $A_x=\{\omega: X(\omega)=x\}, \forall x\in S_X$. These events are both exhaustive and mutually exclusive, hence form a partition. 

%--------------------------------------------------------
% CONTINUOUS RANDOM VARIABLES
%--------------------------------------------------------

\subsection{DEFINITION \textcolor{OrangeRed3} {Continuous Random Variable}}

\textsc{A \textcolor{red}{continuous random variable} is one for which the set of possible values $S_X$ is \textbf{uncountable}. That is, $X$ can take only a countable number of values.}

\begin{itemize}
	\item $F_X$ is continuous
	\item $S_X$ is uncountable
	\item $P(X=x)=0$ for every $x\in\R$ so the probability mass function is not useful.
	\item We assign probabilities to intervals. These probabilities can be obtained from the Cumulative Distribution Function.
	\item $P(a<X<b)=P(a\leq X \leq b)$ \\ \textcolor{SpringGreen4}{Note: This is because $P(X=x)=0$ so adding $P(X=a)$ or $P(X=b)$ to the inequality does not affect the probability.}
\end{itemize}

\textcolor{SpringGreen4}{\textbf{Note:} We will only really look at "absolutely continuous" distributions, not at Cantor's function for instance.}


%--------------------------------------------------------
%--------------------------------------------------------
%--------------------------------------------------------


\section{From Sample Space to Distributions}

\begin{enumerate}
	\item You have a sample space $\Omega$ containing outcomes $\omega\in\Omega$
	\item Events are sets of outcomes: $A \subseteq \Omega$
	\item The probability measure $P$ maps events to $[0,1]$
	\item A random variable maps outcomes in $\Omega$ to $\R$
	\item The \textcolor{red}{cumulative distribution function} $F_X$ maps\\$\R$ to $[0,1]$:~
	
	$\omega\in\Omega $\\ $\Rightarrow X: \Omega \rightarrow \R \;$
	\textcolor{SpringGreen4}{(i.e. $x = X(\omega):\omega\in\Omega$)}\\ 
	$\Rightarrow  F_X: \R \rightarrow [0,1] $
	\textcolor{SpringGreen4}{(i.e. $F_X(x)=P(\{\omega:X(\omega)\leq x\})$)}
	
	\item For a discrete random variable the \textcolor{red}{probability 
			mass function} maps the set of possible values $S_X$ to $[0,1]$
	
	\item For a continuous random variable the \textcolor{red}{probability 
			density function} maps $S_X\subseteq\R$ to $[0,\infty)$
\end{enumerate}



%--------------------------------------------------------
%--------------------------------------------------------
%--------------------------------------------------------

\section{Distribution Function}

\subsection{Properties of the Distribution Function}

\begin{enumerate}
	\item $0\leq F_X(x)\leq 1$,
	\item $P(a<X\leq b) = F_X(b)- F_X(a)$, if $a<b$,
	\item $F_X(x)$ is non-decreasing, \textcolor{red}{*}
	\item $F_X(-\infty)=0, F_X(\infty)=1$. \textcolor{red}{*}
	\item $F_X(\cdot)$ is "right-continuous", this is, for every $x \in \R$, $lim_{h\downarrow 0}F_X(x+h) = F_X(x)$. \textcolor{red}{*}
	\item $P(X=x)$ is the jump in $F_X$ at $x$. That is $P(X=x)=F_X(x)-lim_{h\downarrow 0}F_X(x-h)$
\end{enumerate}

\textcolor{red}{*} Any function $F: \R \rightarrow \R$ satisfying properties $3, 4, 5$ is in fact a distribution function.


\begin{center}
\begin{tabular}{| c | c |}
 \hline
 \textbf{Discrete} & \textbf{Continuous}  \\ 
 \hline
 \textcolor{red}{pmf} \textcolor{violet}{$p_X(x)$} & \textcolor{red}{pdf} \textcolor{violet}{$f_X(x)$} \\
 \hline
 prob. masses & no positive masses \\
 \textcolor{violet}{$p_X(x)$} & \textcolor{violet}{$P(X=x)=0 \forall x$} \\
 \hline
 \textcolor{violet}{$\Sigma_{x\in S_X} p_X(x)=1$} &  \textcolor{violet}{$\int_{-\infty}^{\infty} f_X(t)dt=1$}\\
 \hline
 \textcolor{violet}{$P(X\in I) = \Sigma_{x\in I} p_X(x) $} &  \textcolor{violet}{$P(X\in I)=\int_{a}^{b} f_X(t)dt$}\\
 \hline
 \textcolor{violet}{$0\leq p_X(x) \leq 1$} &  \textcolor{violet}{$f_X(x)\geq0$}\\
 \hline
\end{tabular}
\end{center}

where \textcolor{violet}{$I=[a,b]$}

%--------------------------------------------------------
%--------------------------------------------------------
%--------------------------------------------------------

\section{Expectation \& Variance}

\subsection{Expectation}
\begin{equation}
	\textcolor{violet}{E(X) = \sum_{x\in S_X} x \cdot p_X(x)}
\end{equation}

\begin{equation}
	\textcolor{violet}{E(X) = \int_{-\infty}^{\infty} x f_X(x) dx}
\end{equation}

Accounting trick:

\begin{equation}
	\textcolor{violet}{E(X) = \sum_{all\; \omega \in \Omega} X(\omega) P(\{\omega\})}
\end{equation}

Expectation of functions:

\begin{equation}
	\textcolor{violet}{E(\psi(X)) = \sum_{all\; \omega \in \Omega} \psi(\omega) p_X(x)}
\end{equation}

\begin{equation}
	\textcolor{violet}{E(\psi(X)) = \int_{-\infty}^{\infty} \psi(x) f_X(x) dx}
\end{equation}


\subsection{Variance}

\begin{equation}
	\textcolor{violet}{V(X)= E((X-E(X))^2)}
\end{equation}

If $E(X)$ is finite then:

\begin{equation}
	\textcolor{violet}{V(X)= E(X^2)-E(X)^2}
\end{equation}

\subsection{Computing Moments via Tail Probabilities}

If $P(X\geq 0)=1$ then, for $n\geq 0$:

\begin{equation}
	\textcolor{violet}{E(X^n)=n\int_0^\infty x^{n-1}(1-F_X(x))dx}
\end{equation}

Note: \textcolor{violet}{$1-F(x)=P(X>x)=\int_x^\infty f(t)dt$}

%--------------------------------------------------------
%--------------------------------------------------------
%--------------------------------------------------------


\section{Special Probability Distributions}

\subsection{Discrete Distributions}

%--------------------------------------------------------

\subsubsection{Bernoulli}

\begin{itemize}
	\item \textbf{\textcolor{red}{pmf}}: \[ \begin{cases} 
      1-p & 0 \\
      p & 1 \\
   \end{cases}
\]
	\item \textcolor{red}{mean}: \textcolor{violet}{$E(X)=p$}
	\item \textcolor{violet}{$E(X^2)=p$}
	\item \textcolor{red}{variance}: \textcolor{violet}{$p(1-p)$}
	\item \textcolor{blue}{Binomial, Geometric, Negative Binomial} all arise in the context of a \textcolor{red}{sequence of independent Bernoulli trials}.
\end{itemize}

%--------------------------------------------------------

\subsubsection{Binomial}

\begin{itemize}
	\item Number of "successes" in $n$ Bernoulli Trials
	\item \textbf{\textcolor{red}{pmf}}: \textcolor{violet}{$p_X(x)=P(X=x)={n\choose x}p^x (1-p)^{n-x},\; x=0,1,2,...n$}
	\item Write $X \sim \;$Bin$(n,p)$
	\item \textcolor{red}{mean}: \textcolor{violet}{$E(X)=np$}
	\item \textcolor{violet}{$E(X(X-1)=n(n-1)p^2$}
	\item \textcolor{red}{variance}: \textcolor{violet}{$V(X)=E(X(X-1))+E(X)-E(X)^2$\\$=np(1-p)$.}
\end{itemize}

%--------------------------------------------------------

\subsubsection{Geometric}

\begin{itemize}
	\item Number of ``failures'' before first ``success'' in sequence of Bernoulli Trials
	\item \textbf{\textcolor{red}{pmf}}: \textcolor{violet}{$p_N(n)=P(N=n)=(1-p)^n p\;,\;\; n=0,1,2,...$}
	\item Write $X \sim \; $Geom$(p)$
	\item \textcolor{red}{mean}: \textcolor{violet}{$E(X)=\frac{1-p}{p}$}

	\item \textcolor{red}{variance}: \textcolor{violet}{$V(X)=\frac{1-p}{p^2}$.}
\end{itemize}


%--------------------------------------------------------

\subsubsection{Negative Binomial}

\begin{itemize}
	\item Number of ``failures'' before $r$-th ``success'' in sequence of Bernoulli Trials
	\item \textbf{\textcolor{red}{pmf}}: \textcolor{violet}{$p_Z(z)=P(Z=z)={-r \choose z}p^r (p-1)^z \;,\;\; z=0,1,2,...$}
	
	Where \textcolor{violet}{$r>0$, $0<p\leq 1$}
	
	And \textcolor{violet}{${-r \choose z} = (-1)^z \;{{z+r-1} \choose {r-1}}$}
	\item Write $Z \sim \; $Nb$(r,p)$
	\item \textcolor{red}{mean}: \textcolor{violet}{$E(Z)=\frac{r(1-p)}{p}$}

	\item \textcolor{red}{variance}: \textcolor{violet}{$V(Z)=\frac{r(1-p)}{p^2}$.}

	\item \textcolor{red}{recursive formula}: \textcolor{violet}{\textbf{r}$(z)=\frac{p_{Z_r}(z)}{p_{Z_r}(z-1)}=(\frac{r-1}{z}+1)(1-p), \; z=1,2,...$}

\end{itemize}


%--------------------------------------------------------

\subsubsection{Hypergeometric}

\begin{itemize}
	\item Choose \textcolor{violet}{$m$} samples from a population of \textcolor{violet}{$n$}, where \textcolor{violet}{$d \;(d<n)$} items are defective. What is the probability of getting \textcolor{violet}{$x$} defective samples?
	\item \textbf{\textcolor{red}{pmf}}: \textcolor{violet}{$p_X(x)=P(X=x)=\frac{{d \choose x} {n-d \choose m-x}}{{n \choose m}} \;,\;\; x=a,...,b$}
	
	\item Write $X \sim \; $Hg$(m,d,n)$
	\item \textcolor{red}{mean}: \textcolor{violet}{$E(X)=\frac{m d}{n}$}

	\item \textcolor{red}{variance}: \textcolor{violet}{$V(X)=\frac{md(n-d)}{n^2}\times (1-\frac{m-1}{n-1})$.}

\end{itemize}


%--------------------------------------------------------

\subsubsection{Poisson}

\begin{itemize}
	\item Choose \textcolor{violet}{$m$} samples from a population of \textcolor{violet}{$n$}, where \textcolor{violet}{$d \;(d<n)$} items are defective. What is the probability of getting \textcolor{violet}{$x$} defective samples?
	\item \textbf{\textcolor{red}{pmf}}: \textcolor{violet}{$p_X(x)=P(X=x)=\frac{{d \choose x} {n-d \choose m-x}}{{n \choose m}} \;,\;\; x=a,...,b$}
	
	\item Write $X \sim \; $Hg$(m,d,n)$
	\item \textcolor{red}{mean}: \textcolor{violet}{$E(X)=\frac{m d}{n}$}

	\item \textcolor{red}{variance}: \textcolor{violet}{$V(X)=\frac{md(n-d)}{n^2}\times (1-\frac{m-1}{n-1})$.}

\end{itemize}


%--------------------------------------------------------
%--------------------------------------------------------
%--------------------------------------------------------

% ORIGINAL STUFF:

%--------------------------------------------------------
%--------------------------------------------------------
%--------------------------------------------------------






\pagebreak

\section{Math vs. text vs. functions}
In properly typeset mathematics  variables appear in italics (e.g., $f(x)=x^{2}+2x-3$). The exception to this rule is predefined functions (e.g., $\sin (x)$). Thus it is important to \textbf{always} treat text, variables, and functions correctly. See the difference between $x$ and x, -1 and $-1$, and $sin(x)$ and $\sin(x)$.  

There are two ways to present a mathematical expression--- \emph{inline} or as an \emph{equation}.

\subsection{Inline mathematical expressions}
Inline expressions occur in the middle of a sentence.  To produce an inline expression, place the math expression between dollar signs (\verb!$!).  For example, typing \verb!$90^{\circ}$ is the same as $\frac{\pi}{2}$ radians!  yields $90^{\circ}$ is the same as $\frac{\pi}{2}$ radians.

\subsection{Equations}
Equations are mathematical expressions that are given their own line and are centered on the page.  These are usually used for important equations that deserve to be showcased on their own line or for large equations that cannot fit inline. To produce an inline expression, place the mathematical expression  between the symbols  \verb!\[! and \verb!\]!. Typing \verb!\[x=\frac{-b\pm\sqrt{b^2-4ac}}{2a}\]! yields \[x=\frac{-b\pm\sqrt{b^2-4ac}}{2a}.\]
 
\subsection{Displaystyle} 
To get full-sized inline mathematical expressions  use  \verb!\displaystyle!. Use this sparingly. Typing \verb!I want this $\displaystyle \sum_{n=1}^{\infty}! \verb!\frac{1}{n}$, not this $\sum_{n=1}^{\infty}! \verb!\frac{1}{n}$.! yields\\ I want  this $\displaystyle \sum_{n=1}^{\infty}\frac{1}{n}$, not this $\sum_{n=1}^{\infty}\frac{1}{n}.$


\section{Images}

You can put images (pdf, png, jpg, or gif) in your document. They need to be in the same location as your .tex file when you compile the document. Omit   \verb![width=.5in]! if you want the image to be full-sized.

\verb!\begin{figure}[ht]!\\
\verb!\includegraphics[width=.5in]{imagename.jpg}!\\
\verb!\caption{The (optional) caption goes here.}!\\
\verb!\end{figure}!

\subsection{Text decorations}

Your text can be \textit{italics} (\verb!\textit{italics}!), \textbf{boldface} (\verb!\textbf{boldface}!), or \underline{underlined} (\verb!\underline{underlined}!).

Your math can contain boldface, $\mathbf{R}$ (\verb!\mathbf{R}!), or blackboard bold, $\mathbb{R}$ (\verb!\mathbb{R}!). You may want to used these to express the sets of real numbers ($\mathbb{R}$ or $\mathbf{R}$), integers ($\mathbb{Z}$ or $\mathbf{Z}$), rational numbers ($\mathbb{Q}$ or $\mathbf{Q}$), and natural numbers ($\mathbb{N}$ or $\mathbf{N}$).

To have text appear in a math expression use \verb!\text!. \verb!(0,1]=\{x\in\mathbb{R}:x>0\text{ and }x\le 1\}! yields $(0,1]=\{x\in\mathbb{R}:x>0\text{ and }x\le 1\}$. (Without the \verb!\text! command it treats ``and'' as three variables: $(0,1]=\{x\in\mathbb{R}:x>0 and x\le 1\}$.)



\section{Spaces and new lines}

\LaTeX\ ignores extra spaces and new lines. For example, 

\verb!This   sentence will       look!

\verb!fine after      it is     compiled.!

This   sentence will       look
fine after      it is     compiled.


Leave one full empty line between two paragraphs. Place \verb!\\! at the end of a line to create a new line (but not create a new paragraph).

\verb!This!

\verb!compiles!

~

\verb!like\\!

\verb!this.!

This
compiles 

like\\
this.

Use  \verb!\noindent! to prevent a paragraph from indenting.

\section{Comments}

Use \verb!%! to create a comment. Nothing on the line after the \verb!%! will be typeset. \verb!$f(x)=\sin(x)$ %this is the sine function! yields $f(x)=\sin(x)$%this is the sine function

\section{Delimiters}

\begin{tabular}{lll}
\emph{description} & \emph{command} & \emph{output}\\
parentheses &\verb!(x)! & (x)\\
brackets &\verb![x]! & [x]\\
curly braces& \verb!\{x\}! & \{x\}\\
\end{tabular}

To make your delimiters large enough to fit the content, use them together with \verb!\right! and \verb!\left!. For example, \verb!\left\{\sin\left(\frac{1}{n}\right)\right\}_{n}^! \verb!{\infty}! produces\\ $\displaystyle \left\{\sin\left(\frac{1}{n}\right)\right\}_{n}^{\infty}$.

Curly braces are non-printing characters that are used to gather text that has more than one character. Observe the differences between the four expressions \verb!x^2!, \verb!x^{2}!, \verb!x^2t!, \verb!x^{2t}! when typeset: $x^2$, $x^{2}$, $x^2t$, $x^{2t}$.


\section{Lists}

You can produce ordered and unordered lists.

\begin{tabular}{lll}
\emph{description} & \emph{command} & \emph{output}\\
unordered list&
\begin{tabular}{l}
\verb!\begin{itemize}!\\
\verb!  \item!\\
\verb!  Thing 1!\\
\verb!  \item!\\
\verb!  Thing 2!\\
\verb!\end{itemize}!
\end{tabular}&
\begin{tabular}{l}
$\bullet$ Thing 1\\
$\bullet$ Thing 2
\end{tabular}\\
~\\
ordered list&
\begin{tabular}{l}
\verb!\begin{enumerate}!\\
\verb!  \item!\\
\verb!  Thing 1!\\
\verb!  \item!\\
\verb!  Thing 2!\\
\verb!\end{enumerate}!
\end{tabular}&
\begin{tabular}{l}
1.~Thing 1\\
2.~Thing 2
\end{tabular}
\end{tabular}


\section{Symbols (in \emph{math} mode)}

\subsection{The basics}
\begin{tabular}{lll}
\emph{description} & \emph{command} & \emph{output}\\
addition & \verb!+! & $+$\\
subtraction & \verb!-! & $-$\\
plus or minus & \verb!\pm! & $\pm$\\
multiplication (times) & \verb!\times! & $\times$\\
multiplication (dot) & \verb!\cdot! & $\cdot$\\
division symbol & \verb!\div! & $\div$\\
division (slash) & \verb!/! & $/$\\
circle plus & \verb!\oplus! & $\oplus$\\
circle times & \verb!\otimes! & $\otimes$\\
equal & \verb!=! & $=$\\
not equal & \verb!\ne! & $\ne$\\
less than & \verb!<! & $<$\\
greater than & \verb!>! & $>$\\
less than or equal to & \verb!\le! & $\le$\\
greater than or equal to & \verb!\ge! & $\ge$\\
approximately equal to & \verb!\approx! & $\approx$\\
infinity & \verb!\infty! & $\infty$\\
dots & \verb!1,2,3,\ldots! & $1,2,3,\ldots$\\
dots & \verb!1+2+3+\cdots! & $1+2+3+\cdots$\\
fraction & \verb!\frac{a}{b}! & $\frac{a}{b}$\\
square root & \verb!\sqrt{x}! & $\sqrt{x}$\\
$n$th root & \verb!\sqrt[n]{x}! & $\sqrt[n]{x}$\\
exponentiation & \verb!a^b! & $a^{b}$\\
subscript & \verb!a_b! & $a_{b}$\\
absolute value & \verb!|x|! & $|x|$\\
natural log  & \verb!\ln(x)! & $\ln(x)$\\
logarithms & \verb!\log_{a}b! & $\log_{a}b$\\
exponential function & \verb!e^x=\exp(x)! & $e^{x}=\exp(x)$\\
degree & \verb!\deg(f)! & $\deg(f)$\\
\end{tabular}
\newpage

\subsection{Functions}
\begin{tabular}{lll}
\emph{description} & \emph{command} & \emph{output}\\
maps to & \verb!\to! & $\to$\\
composition& \verb!\circ! & $\circ$\\
piecewise& \verb!|x|=! & \multirow{5}{*}{$\displaystyle |x|=\begin{cases}x&x\ge 0\\-x&x<0\end{cases}$}\\
function&\verb!\begin{cases}!&\\ 
&\verb!x & x\ge 0\\!&\\ 
&\verb!-x & x<0!&\\ 
&\verb!\end{cases}!&
\end{tabular}

\subsection{Greek and Hebrew letters}
\begin{tabular}{llll}
\emph{command} & \emph{output}&\emph{command} & \emph{output}\\
\verb!\alpha! & $\alpha$&\verb!\tau! & $\tau$\\
\verb!\beta! & $\beta$&\verb!\theta! & $\theta$\\
\verb!\chi! & $\chi$&\verb!\upsilon! & $\upsilon$\\
\verb!\delta! & $\delta$&\verb!\xi! & $\xi$\\
\verb!\epsilon! & $\epsilon$&\verb!\zeta! & $\zeta$\\
\verb!\varepsilon! & $\varepsilon$&\verb!\Delta! & $\Delta$\\
\verb!\eta! & $\eta$&\verb!\Gamma! & $\Gamma$\\
\verb!\gamma! & $\gamma$&\verb!\Lambda! & $\Lambda$\\
\verb!\iota! & $\iota$&\verb!\Omega! & $\Omega$\\
\verb!\kappa! & $\kappa$&\verb!\Phi! & $\Phi$\\
\verb!\lambda! & $\lambda$&\verb!\Pi! & $\Pi$\\
\verb!\mu! & $\mu$&\verb!\Psi! & $\Psi$\\
\verb!\nu! & $\nu$&\verb!\Sigma! & $\Sigma$\\
\verb!\omega! & $\omega$&\verb!\Theta! & $\Theta$\\
\verb!\phi! & $\phi$&\verb!\Upsilon! & $\Upsilon$\\
\verb!\varphi! & $\varphi$&\verb!\Xi! & $\Xi$\\
\verb!\pi! & $\pi$&\verb!\aleph! & $\aleph$\\
\verb!\psi! & $\psi$&\verb!\beth! & $\beth$\\
\verb!\rho! & $\rho$&\verb!\daleth! & $\daleth$\\
\verb!\sigma! & $\sigma$&\verb!\gimel! & $\gimel$
\end{tabular}


\subsection{Set theory}
\begin{tabular}{lll}
\emph{description} & \emph{command} & \emph{output}\\
set brackets & \verb!\{1,2,3\}! & $\{1,2,3\}$\\
element of & \verb!\in! & $\in$\\
not an element of & \verb!\not\in! & $\not\in$\\
subset of & \verb!\subset! & $\subset$\\
subset of & \verb!\subseteq! & $\subseteq$\\
not a subset of & \verb!\not\subset! & $\not\subset$\\
contains & \verb!\supset! & $\supset$\\
contains & \verb!\supseteq! & $\supseteq$\\
union & \verb!\cup! & $\cup$\\
intersection & \verb!\cap! & $\cap$\\
big union & 
\verb!\bigcup_{n=1}^{10}A_n! &
$\displaystyle \bigcup_{n=1}^{10}A_{n}$\\
big intersection & \verb!\bigcap_{n=1}^{10}A_n! &$\displaystyle \bigcap_{n=1}^{10}A_{n}$\\
empty set & \verb!\emptyset! & $\emptyset$\\
power set & \verb!\mathcal{P}! & $\mathcal{P}$\\
minimum & \verb!\min! & $\min$\\
maximum & \verb!\max! & $\max$\\
supremum & \verb!\sup! & $\sup$\\
infimum & \verb!\inf! & $\inf$\\
limit superior & \verb!\limsup! & $\limsup$\\
limit inferior & \verb!\liminf! & $\liminf$\\
closure & \verb!\overline{A}! & $\overline{A}$
\end{tabular}

\subsection{Calculus}
\begin{tabular}{lll}
\emph{description} & \emph{command} & \emph{output}\\
derivative & \verb!\frac{df}{dx}! & $\displaystyle \frac{df}{dx}$\\
derivative & \verb!\f'! & $f'$\\
partial derivative & 
\begin{tabular}{l}
\verb!\frac{\partial f}!\\ \verb!{\partial x}! 
\end{tabular}& $\displaystyle \frac{\partial f}{\partial x}$\\
integral & \verb!\int! & $\displaystyle\int$\\
double integral & \verb!\iint! & $\displaystyle\iint$\\
triple integral & \verb!\iiint! & $\displaystyle\iiint$\\
limits & \verb!\lim_{x\to \infty}! & $\displaystyle \lim_{x\to \infty}$\\
summation  & 
\verb!\sum_{n=1}^{\infty}a_n! &
$\displaystyle \sum_{n=1}^{\infty}a_n$\\
product  & 
\verb!\prod_{n=1}^{\infty}a_n! &
$\displaystyle \prod_{n=1}^{\infty}a_n$
\end{tabular}




\subsection{Logic}
\begin{tabular}{lll}
\emph{description} & \emph{command} & \emph{output}\\
not & \verb!\neg! & $\sim$\\
and & \verb!\land! & $\land$\\
or & \verb!\lor! & $\lor$\\
if...then & \verb!\implies! & $\to$\\
if and only if & \verb!\iff! & $\leftrightarrow$\\
logical equivalence & \verb!\equiv! & $\equiv$\\
therefore & \verb!\therefore! & $\therefore$\\
there exists  & \verb!\exists! & $\exists$\\
for all & \verb!\forall! & $\forall$\\
%implies & \verb!\Rightarrow! & $\Rightarrow$\\
%equivalent & \verb!\Leftrightarrow! & $\Leftrightarrow$
\end{tabular}

\subsection{Linear algebra}
\begin{tabular}{lll}
\emph{description} & \emph{command} & \emph{output}\\
vector & \verb!\vec{v}! & $\vec{v}$\\
vector & \verb!\mathbf{v}! & $\mathbf{v}$\\
norm & \verb!||\vec{v}||! & $||\vec{v}||$\\
matrix&
\begin{tabular}{l}
\verb!\left[!\\
\verb!\begin{array}{ccc}!\\
\verb!1 & 2 & 3 \\!\\
\verb!4 & 5 & 6\\!\\
\verb!7 & 8 & 0!\\
\verb!\end{array}!\\
\verb!\right]!\end{tabular}&
$\displaystyle \left[\begin{array}{ccc}1 & 2 & 3 \\4 & 5 & 6 \\7 & 8 & 0\end{array}\right]$\\
\\determinant&
\begin{tabular}{l}
\verb!\left|!\\
\verb!\begin{array}{ccc}!\\
\verb!1 & 2 & 3 \\!\\
\verb!4 & 5 & 6 \\!\\
\verb!7 & 8 & 0!\\
\verb!\end{array}!\\
\verb!\right|!
\end{tabular}&
$\displaystyle \left|\begin{array}{ccc}1 & 2 & 3 \\4 & 5 & 6 \\7 & 8 & 0\end{array}\right|$\\
determinant & \verb!\det(A)! & $ \det(A)$\\
trace & \verb!\operatorname{tr}(A)! & $\operatorname{tr}(A)$\\
dimension & \verb!\dim(V)! & $\dim(V)$\\
\end{tabular}

\subsection{Number theory}
\begin{tabular}{lll}
\emph{description} & \emph{command} & \emph{output}\\
divides & \verb!|! & $|$\\
does not divide & \verb!\ndv! & $\not |$\\
div & \verb!\dv! & $\operatorname{div}$\\
mod & \verb!\mod! & $\operatorname{mod}$\\
greatest common divisor & \verb!\gcd! & $\gcd$\\
ceiling & \verb!\lceil x \rceil! & $\lceil x\rceil$\\
floor & \verb!\lfloor x \rfloor! & $\lfloor x \rfloor$\\
\end{tabular}




\subsection{Geometry and trigonometry}
\begin{tabular}{lll}
\emph{description} & \emph{command} & \emph{output}\\
angle& \verb!\angle ABC! & $\angle ABC$\\
degree& \verb!90^{\circ}! & $90^{\circ}$\\
triangle& \verb!\triangle ABC! & $\triangle ABC$\\
segment& \verb!\overline{AB}! & $\overline{AB}$\\
sine& \verb!\sin! & $\sin$\\
cosine& \verb!\cos! & $\cos$\\
tangent& \verb!\tan! & $\tan$\\
cotangent& \verb!\cot! & $\cot$\\
secant& \verb!\sec! & $\sec$\\
cosecant& \verb!\csc! & $\csc$\\
inverse sine& \verb!\arcsin! & $\arcsin$\\
inverse cosine& \verb!\arccos! & $\arccos$\\
inverse tangent& \verb!\arctan! & $\arctan$\\
\end{tabular}

\section{Symbols (in \emph{text} mode)}

The followign symbols do \textbf{not} have to be surrounded by dollar signs.

\begin{tabular}{lll}
\emph{description} & \emph{command} & \emph{output}\\
dollar sign & \verb!\$! & \$ \\
percent & \verb!\%! & \% \\
ampersand & \verb!\&! & \& \\
pound & \verb!\#! & \# \\
backslash & \verb!\textbackslash! & \textbackslash \\
left quote marks & \verb!``! & `` \\
right quote marks & \verb!''! & '' \\
single left quote  & \verb!`! & ` \\
single right quote  & \verb!'! & ' \\
hyphen & \verb!X-ray! & X-ray\\
en-dash & \verb!pp. 5--15! & pp. 5--15 \\
em-dash & \verb!Yes---or no?! & Yes---or no? 
\end{tabular}

\section{Resources}
\href{http://www.tug.org/}{TUG: The \TeX\ Users Group}\\
\href{http://www.ctan.org/}{CTAN: The Comprehensive \TeX\ Archive Network}\\
Handwriting-to-\LaTeX\ sites: \href{http://detexify.kirelabs.org/}{Detexify}, \href{http://webdemo.visionobjects.com/equation.html}{WebEquation}\\
\href{ftp://tug.ctan.org/pub/tex-archive/info/symbols/comprehensive/symbols-letter.pdf}{The Comprehensive \LaTeX\ Symbol List}\\ 
\href{http://mirrors.med.harvard.edu/ctan/info/lshort/english/lshort.pdf}{The Not So Short Introduction to \LaTeX\ 2$\varepsilon$}\\
Software that generates \LaTeX\ code: Mathematica, Maple, GeoGebra\\ %\href{https://prep11geogebra.pbworks.com/w/page/38586775/LaTex%20with%20GeoGebra}{Geogebra to \LaTeX\ }\\
\LaTeX\ for the Mac: \href{http://www.tug.org/mactex/}{Mac\TeX}\\
\LaTeX\ for the PC: \href{http://www.texniccenter.org/}{{\TeX}nicCenter} and \href{http://miktex.org/}{MiK\TeX}\\
\LaTeX\ online:  \href{http://www.overleaf.com/}{Overleaf}, \href{http://www.sagemath.org/}{Sage}\\
\LaTeX\ integration with Microsoft Office, Apple iWork, etc: \href{http://www.dessci.com/en/products/mathtype/}{MathType}, \href{http://www.chachatelier.fr/latexit/}{{\LaTeX}{iT}}
\vfill
\hrule
~\\
Dave Richeson, Dickinson College, \href{http://divisbyzero.com/}{http://divisbyzero.com/}
\end{multicols}

\end{document}

